\documentclass[11pt]{article}

\usepackage[pdftex]{graphicx}
\usepackage{amsmath}
\usepackage{algorithm}
\usepackage[noend]{algpseudocode}

\title{UZ Povzetek}
\author{Yannick Kuhar}
\date{\today}

\begin{document}

\clearpage
\maketitle
\thispagestyle{empty}

\newpage

\section{Image formation}

\subsection{Film image}

\v{C}e postavimo objekt pred film ne dobimo dobre slike. Dodamo bariero, ki blokira ve\v{c}ino \v{z}arkov zmanj\v{s}a zameglitev. Odprtina, ki prepu\v{s}\v{c}a \v{z}arke se imenuje \textbf{aperture}(\textit{ang.}). \\
\\
U\v{c}inki velikosti \textbf{apertureja}:
\begin{itemize}
\item \textbf{prevelik}, prepusti preve\v{c} svetlobe kar povzro\v{c}i zameglitev slike
\item \textbf{premajhen}. prepusti premalo svetlobe kar povzro\v{c}i zameglitev slike
\end{itemize}
\
V splo\v{s}nem sta oba ekstrema slaba. Te\v{z}vo povzroca na sploh majhno stevilo zarkov, ki zadanejo film in nastane temna slika. \\
To odpravimo z dadajo le\v{c}e med objekt in film. Le\v{c}a fokusira \v{z}arke na film. To\v{c}ke na nekaterih razdalijah ostanejo fokusirane na nekatih pa zamegljene. \\
Pri uporabi tanke le\v{c}e, to\v{c}ke na razlicnih globinah fokusirajo na razlicnih globinah slikovne raznine(\textit{ang.} \textbf{image plane}). \\
\\
\textbf{Depth of field} zardalija med \textbf{slikovnim ravninam} med katerimi je u\v{c}inek zameglitve dovolj majhen. Majhni \textbf{aperture} pove\v{c}a \textbf{depth of field}. \\
\\
\textbf{Field of view}(\textbf{FOV}) ($2 \times \varphi$) je kotna mera prostora, ki ga zazna kamera. \\
\\
\indent $\varphi = tan^{-1}(\frac{d}{2f})$\\
\\
\textbf{Field of view}:
\begin{itemize}
\item majhen \textbf{focal length}($f$) prinese \v{s}irokokotno sliko
\item visoko \textbf{focal length}($f$) prinese teleskopsko sliko
\end{itemize}
\
\textbf{Aberacije}:
\begin{itemize}
\item \textbf{Chromatic aberration}, razli\v{c}ne valovne dolzine se lomijo pod razli\v{c}nimi koti in fokusirajo pod razli\v{c}nimi razdalijami
\item \textbf{Spherical aberration}, sferi\v{c}ne le\v{c}e ne fokusirajo svetlobo perfektno, \v{z}arki blizje robu se fokusirajo bli\v{z}je kot \v{z}arki pri centru
\end{itemize}

\newpage

\textbf{Vignetting} se zgodi, ko imamo ve\v{c} kot eno le\v{c}o v kameri. Fokusirani \v{z}arki iz prve(proti centru) le\v{c}e se lomijo preko druge. Ker so \v{z}arki iz prve usmerjeni proti centru se namo prseslikajo preko druge(\v{z}arki pod kotom se filtriranjo). To povzro\v{c}i temne kote.\\
\\
Pogosta nepravilnost je tudi t.i. \textbf{Radial distortion}(\textit{ang.}), ki se zgodi zaradi neprefektnosti le\v{c}.

\subsection{Digital image}
Namesto filma uporabimo matriko(tabelo) senzorjev. Diskretiziramo sliko v piksle in kvantiziramo svetlobo v intenzitetne nivoje.

\subsection{CCD vs CMOS}
\begin{itemize}
\item v obeh fotoni povzro\v{c}ijo naboj na senzorjih
\item \textbf{CCD} prebere naboj po principu \textbf{FIFO} in digitalizira
\item \textbf{CMOS} digitalizira vsak piksel posebaj
\item \textbf{CCD} slike so bolj\v{s}e kvalitete
\item \textbf{CMOS} je cenej\v{s}i za izdelavo
\end{itemize}
\
Zaradi osvetljenosti, ki je dolo\v{c}ena s strani zelene barve, uporabimo dvakrat toliko zelenih senzorjev kot modrih in rde\v{c}ih. \v{C}love\v{s}ki vid je bolj ob\v{c}utljiv na spremembe intenzitete kot na barvne spremembe(\textbf{Bayer sensor}).

\newpage

\section{Image processing 1}
Binarne slike imajo samo dve mo\v{z}nosti intenzitete:
\begin{itemize}
\item \textbf{ospredje}(1)
\item \textbf{ozadje}(0)
\end{itemize}

\section{Image tresholding}
Spremenimo sliko v binarno masko. Spoznali smo tri na\v{c}ine:
\begin{itemize}
\item \textbf{single threshold} $F_T[i, j] = $
$
\begin{cases}
	1, & \text{if } F[i, j] \geq T \\
	0, & \text{sicer}
\end{cases}
$
\item \textbf{two thresholds} $F_T[i, j] = $
$
\begin{cases}
	1, & \text{if } T_1 \leq F[i, j] \leq T_2 \\
	0, & \text{sicer}
\end{cases}
$
\item \textbf{apply a classifier} $F_T[i, j] = $
$
\begin{cases}
	1, & \text{if } F[i, j] \in Z \\
	0, & \text{sicer}
\end{cases}
$
\end{itemize}
\
Binarne slike lahko opi\v{s}emo z \textbf{bimodalnim histogramom}(ima le dva stolpca in predstavlja porazdelitev pikslov).

\subsection{Globalna binarizacija}
Najdi tak\v{s}en $T$, ki minimizira varianco med razredi, katere $T$ lo\v{c}uje. \\
\\
\indent $\sigma^2_{within}(T) = n_1(T)\sigma^2_1 + n_2(T)\sigma^2_2$ \\
\\
\indent $n_1(T) = |\{I_{(x, y)} < T\}|$, $n_2(T) = |\{I_{(x, y)} \geq T\}|$ \\
\\
Oziroma druga\v{c}e maksimiziramo vmesen razred. \\
\\
\indent $\sigma^2_{between}(T) = \sigma^2 - \sigma^2_{within}(T) = n_1(T)n_2(T)[\mu_1(T) - \mu_2(T)]^2$

\newpage

\subsection{Otsujev algoritem}

\begin{algorithm}
\caption{Otsu's algorithm}\label{euclid}
\begin{algorithmic}[1]

\Procedure{Otsu}{I}

\State $\textit{nbins} \gets 256$
\State $counts \gets myhist(I, nbins)$ \% Get histohram
\State $\textit{p} \gets counts / sum(counts)$ \% Normalize the histohram
\State $\sigma \gets zeros(nbins, 1)$

\For{t = 1 : nbins}
\State $qlow \gets sum(p(1:t))$
\State $qhigh \gets sum(p(t + 1 : end))$
\State $\mu_L \gets sum(p(1:t)$ $.*$ $(1:t)) / qlow$ 
\State $\mu_H \gets sum(p(t + 1 : end)$ $.*$ $(t + 1 : nbins))$ $./$ $qhigh$
\State $\sigma(t) \gets qlow * qhigh * (\mu_L - \mu_H) ^ 2$
\EndFor

\State $[\sim, threshold] \gets max(\sigma(:))$

\Return threshold

\EndProcedure
\end{algorithmic}
\end{algorithm}

\subsubsection*{Neformalno}
Za vsak \textbf{treshold T} razdeli sliko na dva razreda. Izra\v{c}unaj $\sigma$ vmesnega razreda in jo shrani. Vrni tak $\sigma$, ki maksimizira razpr\v{s}enost vmesnega razreda.

\subsection{Lokalna binarizacija}
Ocenimo lokalni \textbf{treshold T} na oknu velikosti \textbf{W}: \\
\\
\indent $T_W = \mu_W + k \times \sigma_W$ \\
\\
In to storimo na vsaki regiji. Postopek se imenuje \textbf{Niblack's algorithm}.

\newpage

\section{\v{C}i\v{s}\v{c}enje slik}
\textbf{Tresholding} ne odstrani \v{s}uma z slik, zato potrebujemo dodatno procesiranje. To storimo z morfolo\v{s}kimi operatorji.

\subsection{Morfologija}

\subsubsection*{Structuring element}
Je vsebina premikajo\v{c}ega okna s katerim se pomikamo po sliki. Na primer: \\
 \[
   SE =
  \left[ {\begin{array}{ccccccccc}
   0 & 1 & 0 \\
   1 & 1 & 1 \\
   0 & 1 & 0 \\
  \end{array} } \right]
\]

\subsubsection*{Fitting \& Hitting}
\begin{itemize}
\item \textbf{Fit}, SE in regija imasta vse "1" istole\v{z}ne
\item \textbf{Hit}, SE in regija imasta vsaj eno istole\v{z}no "1"
\end{itemize}

\subsubsection*{Erosion}
Funkcijo slike \textbf{f} in \textbf{structuring element s} ozna\v{c}imo z $g = f \ominus s$
oziroma: \\
\\
\indent $g(x, y) = $
$
\begin{cases}
	1, & \text{if } s \textbf{ fits } f \\
	0, & \text{sicer}
\end{cases}
$

\subsubsection*{Dilation}
Funkcijo slike \textbf{f} in \textbf{structuring element s} ozna\v{c}imo z $g = f \oplus s$
oziroma: \\
\\
\indent $g(x, y) = $
$
\begin{cases}
	1, & \text{if } s \textbf{ hits } f \\
	0, & \text{sicer}
\end{cases}
$

\subsubsection*{Combined operations}
\begin{itemize}
\item \textbf{Opening}, erosion in dilation, odstrani majhne objekte in ohrani grobe oblike
\item \textbf{Closing}, dilation in erosion, napolni luknje in ohrani originalno obliko
\end{itemize}

\newpage

\section{Ozna\v{c}evanje regij}
Procesiramo sliko od leve proti desni, od zgoraj dol:
\begin{enumerate}
\item \v{C}e je vrednost trenutnega piksla je 1
\begin{enumerate}
\item \v{c}e je zgornji ali levi sosed 1 kopiraj oznako
\item \v{c}e sta oba zgornji in levi 1 in imata enako oznako jo kopiraj
\item \v{c}e imata razlicne oznake kopiraj levo in posidobi tabelo ekvivalentnih oznak
\item sicer uporabi novo oznako
\end{enumerate}
\item ponovno ozna\v{c}i z najmanjsim stevilom ekvivalentnih oznak
\end{enumerate}

\section{Region descriptors}
\v{Z}elimo tak\v{s}en \textbf{descriptor}, ki slika dve sliki z podobnim objektom blizu in z razli\v{c}nim objektom dale\v{c}.

\section{Image processing 2}

\subsection{Svetloba}
Svetloba je elektromagnetska radiacija sestavljena iz ve\v{c} frekvenc. Lastnosti svetlobe so opisane v njenem spektru. Ljudje prepoznavamo svetlobo s pomo\v{c}jo deh tipov celic \textbf{cones} za barve(\textbf{R, G, B}) in \textbf{rods} za intenziteto.

\subsection{Additive mixture model}
Je klasi\v{c}en \textbf{RGB} sistem. Kjer \v{c}rni barvi dodajamo rde\v{c}o, modro in zeleno.

\subsection{Subtractive models}
Je \textbf{CMYK} model(\textbf{cyan, magenta, yellow, in key}).

\newpage

\subsection{Color spaces}
Omogo\v{c}a reprodukcijo barv. Vsak prostor ima primarne barve in vsaka nova barva je utežena vsota primarnih barv. Razdalije med barvbami so evklidske.

\subsection{HSV colorspace}
Hue (barvnost), saturation (nasičenje), value (intenziteta), je nelinearen barvni prostor.

\section{Podobnost barv z histogrami}
Histogram je stolpi\v{c}ni diagram, ki meri nivoje inteanzitete npr: \\
\\
\indent $h(i) =$ \v{s}t. pikslov v sliki z intenziteto $i$. \\
\indent $h(i) = card\{(u,v)$ $|$ $I(u,v) = i\}$ (\textbf{card} je funkcija, ki oceni vrednost) \\
\\
Lahko imamo tudi barvne histograme: \\
\\
\indent $h(r, g, b) =$ \v{s}t. pikslov v sliki z barvo (r, g, b).

\subsection{Normalizacija intenzitete}
Intenziteta je vsebovana v vsakem brvnem kanalu. Mno\v{z}enje bsrve z skaljarjem spremeni intenziteto ampak ne barve. To pomeni, da lahko nozmaliziramo intenziteto(\textbf{I = R + B + G}) barve. Kromatska reprezentacija: \\
\\
\indent $r = \frac{R}{R+B+G}$, $b = \frac{B}{R+B+G}$, $g = \frac{G}{R+B+G}$, $r + b + g = 1$ \\
\\
Zdaj lahko uporabimo 2D prostor(rg). Lahko pa tudi indirektno primerjamo slike tako, da primerjamo njihove histograme(Evklidska razdalija, Hellingerjeva razdalija, ...).

\newpage

\section{Filtriranje}
Glavna naloga je reduciranje \v{s}uma v sliki.

\subsection{Tipi \v{s}uma}
\begin{itemize}
\item \textbf{Sol in poper}, naklju\v{c}ne bele in \v{c}rne to\v{c}ke na sliki
\item \textbf{Impulzivni \v{s}um} naklju\v{c}ne bele to\v{c}ke na sliki
\item \textbf{Gaussov \v{s}um} variacija intenzitete je vzor\v{c}ena normalno
\end{itemize}

\subsection{Poskusimo odstranit piksle}
Piksli so podobni kot sosednji. Predpostavimo, da je \v{s}um \\ 
\textbf{I.I.D(independent, identically distributed)}. \\
Sedaj lahko izra\v{c}unamo bolj\v{s}o oceno intenzitete piksla tako, da ga zamenjamo z povpre\v{c}no intenziteto soseske pikslov. Predpostavimo, da imajo vsi piksli enako ute\v{z} in soseska velikosti 2k+1: \\
\\
\indent $G[i, j] = \frac{1}{(2k+1)^2}\sum\limits_{u=-k}^k\sum\limits_{v=-k}^kF[i+u, j+v]$ \\
\\
Oziroma bolj splo\v{s}no, \v{c}e dodamo ute\v{z}i, ki niso enake: \\
\\
\indent $G[i, j] = \sum\limits_{u=-k}^k\sum\limits_{v=-k}^kH[u, v]F[i+u, j+v]$ \\
\\
Kjer so $H[u, v]$ neenake ute\v{z}i. \\
\\
Temu se re\v{c}e \textbf{cross-correlation} oz. \textbf{correlation filtering}($G = H \otimes F$), $H$ je okno in $F$ je slika.

\subsection{Konvolucija kot korelacija}
Izra\v{c}unamo konvolucijo z \textbf{cross-correlation}. Obrnemo filter v obe dimenzijah(horizontalno in vertikalno) in nato izvedemo \textbf{cross-correlation} in to ozna\v{c}imo $G = H \star F$ oziroma: \\
\\
\indent $G[i, j] = \sum\limits_{u=-k}^k\sum\limits_{v=-k}^kH[u, v]F[i-u, j-v]$

\newpage

\subsection{Convolution vs. Correlation}
\begin{itemize}
\item \textbf{Correlation} $G[i, j] = \sum\limits_{u=-k}^k\sum\limits_{v=-k}^kH[u, v]F[i+u, j+v]$, $G = H \otimes F$
\item \textbf{Convolution} $G[i, j] = \sum\limits_{u=-k}^k\sum\limits_{v=-k}^kH[u, v]F[i-u, j-v]$,
$G = H \star F$
\end{itemize}
\
Za simetri\v{c}en filter $H[-u, -v] = H[u, v]$ velja  \textbf{correlation} $\equiv$ \textbf{convolution}

\subsection{Linearni shift-invariant sistem}
\textbf{Shift-invariant} pomeni enako obna\v{s}anje ne glede na pozicijo oz. izhod je odvisen le od lokalnega vzorca ne pa od koordinat.

\subsection{Lastnosti konvolucije}
\begin{itemize}
\item linearnost, shift-invariant
\item komutativnost: $f * g = g * f$
\item asociativnost: $(f * g) * h = f * (g * h)$, posledica tega je da uporaba ve\v{c}filtrov je ekvivalentna uporabi enega $((f * b_1) * b_2) * b_3 = f * (b_1 * b_2 * b_3)$
\item identiteta: enotni impulz $e = [..., 0, 0, 1, 0, 0, ...]$
\item odvod: $\frac{\partial}{\partial x}(f * g) = (\frac{\partial}{\partial x}g) * f = (\frac{\partial}{\partial x} * f) * g$
\end{itemize}

\subsection{Gaussovo glajenje}
Uporabimo \textbf{Gaussian kernel}: \\
\\
\indent $G_\sigma = \frac{1}{2\pi\sigma^2}e^{-\frac{(x^2 + y^2)}{2\sigma^2}}$ \\
\\
Piksli pri centru imajo ve\v{c}jo ute\v{z}. Filter je simetri\v{c}en glede na rotacijo(transponiranje ne spremeni ni\v{c}esar). Varianca $\sigma^2$ dolo\v{c}i mero glajenja.

\newpage

\subsection{U\v{c}inkovita implementacija}
Izvedemo konvolucijo nad vsako vrstico posebaj z 1D kernelom: \\
\\
\indent $g(x) = \frac{1}{\sqrt{2\pi}\sigma}exp(-x^2 / (2\sigma^2))$ \\
\\
Nato izvedemo 1D konvolucijo nad vsakim stolpcem: \\
\\
\indent $g(y) = \frac{1}{\sqrt{2\pi}\sigma}exp(-y^2/(2\sigma^2))$ \\
\\
Ta lo\v{c}itev je mogo\v{c}a zaradi linearnosti, asociativnosti in komutativnosti konvolucije oz: \\
\\
\indent $g_x * (g_y * I) = (g_x * g_y) * I$

\subsection{Robni pogoji filtriranja}
Kernel presega robove slike. Kaj storimo na robovih? Uporabimo \v{s}tiri metode:
\begin{itemize}
\item \textbf{Crop}(black)
\item \textbf{Bend} image
\item \textbf{Replicate} edges
\item \textbf{Mirror} image
\end{itemize}

\subsection{Konvolucija in spekter}
Konvolucija dveh funkcij je ekvivalentna produktu njunih Fourierjevih transformacij(spekter): \\
\\
\indent $F(f * g) = F(f) \odot F(g)$ \\
\\
Konvolucija manipulira slikovni spekter bodisi poudari bodisi zmanj\v{s}a frekven\v{c}ni pass v sliki.

\newpage

\subsection{Ponovitev Fourierjeve transformacije}
Signal je definiran ko vsota sinusov in cosinusov razli\v{c}nih frekvenc: \\
\\
\indent $f(x) = \sum\limits_n a_ncos(nx) + b_nsin(nx)$ \\
\\
\v{S}um je v resnici prisotnost visokih frekvenc. Odstranimo ga tako, da izvedemo t.i. \textbf{low-band pass} filter. Uporabimo lahko tudi t.i. \textbf{spatial box} filter vendar ta ima stranske u\v{c}inke. Najbolj primeren \textbf{low-band pass} filter je \textbf{Gaussov}.

\subsection{Sharpening filter}
$I * (
  \left[ {\begin{array}{ccccccccc}
   0 & 0 & 0 \\
   0 & 2 & 0 \\
   0 & 0 & 0 \\
  \end{array} } \right]
$ - $\frac{1}{9}$
$
  \left[ {\begin{array}{ccccccccc}
   1 & 1 & 1 \\
   1 & 1 & 1 \\
   1 & 1 & 1 \\
  \end{array} } \right]
) = I_s$ \\
\\
$I_s$ je izostrena slika. Sharpening filter poudari razlike z lokalnim povpre\v{c}enjem.

\subsection{Median filter}
Je nelinearen filter. Osnovna ideja je zamenjati intenziteto piksla z mediano drse\v{c}ega okna. Pomembni lastnosti sta:
\begin{itemize}
\item Ne dodaja novih sivih nivojev v sliko
\item odstrani impulzne \v{s}ume npr. \textbf{salt \& pepper}
\end{itemize}

\newpage

\subsection{Filtering as template matching}
Z metodami filtriranja lahko tudi i\v{s}\v{c}emo lokacio objekta na sliki. Template je lahko ve\v{c}ji ali manj\v{s}i kot slika zato moramo izvesti korelacijo za razli\v{c}ne velikosti slike. Vendar namesto templatea skaliramo vhod. U\v{c}inkovit na\v{c}in spreminjanja velikosti slik je t.i. \textbf{Slikovna piramida}.

\subsubsection{Kako zmanj\v{s}amo sliko?}
Naivni pristop je odstranitev vsakega drugega piksla. U\v{c}inku pravimo \textbf{Aliasing}. Pristo povzro\v{c}a te\v{z}ave saj spremeni objekte na sliki.

\subsubsection{Prekrivanje (sampling and aliasing)}
Zvezni signal diskretiziramo, izvedemo Fourierjevo transformacijo. Iz magnitudnega spektra izberemo eno kopijo in jo pretvorimo nazaj z inverzno Fourierjevo transformacijo(slika na Image processing 2 slide 68). \v{C}e \v{z}elimo rekonstruirat vse frekvence do $f$ potem moramo vzor\v{c}it signale z frekvenco enako vsaj $2f$. Na neki to\v{c}ki se za\v{c}nejo ponavljajo\v{c}e se frekvence spekrta prekrivat(aliasing).

\subsubsection{Pravilno vzor\v{c}enje}
Ko izvedemo t.i. sparse sampling moramo odstranit vsebino z visoko frekvnco. To storimo z Gaussovim filtrom. Na ta na\v{c}in ponavljajo\v{c}i se spektri vzor\v{c}enega signala se ne prekrivajo in ne poudarjajo napa\v{c}nih frekvenc. Tako prepre\v{c}imo aliasing.

\newpage

\subsection{Gaussova piramida}
$G_0 = Image$ \\
$G_1 = (G_0 * gaussian) \downarrow 2$ (smooth then subsample) \\
$G_2 = (G_1 * gaussian) \downarrow 2$ (smooth then subsample) \\
$G_3 = (G_2 * gaussian) \downarrow 2$ (smooth then subsample) \\
$G_4 = (G_3 * gaussian) \downarrow 2$ (smooth then subsample) \\
... \\
\\
Nov nivo dobimo od prej\v{s}njega tako, da gladimo z Gaussovim filtrom in naro vzor\v{c}imo. Gaussov filter uporabimo zarai naslednje lastnosti: \\
\\
$G(\sigma_1^2) *G(\sigma_2^2) = G(\sigma_1^2 + \sigma_2^2)$ \\
\\
Zelo dobro deluje z konvolucijo. Poleg tega je t.i. \textbf{low-band-pass filter}, zato dobimo nepotrebno reprezentcijo zgladene slike. Ni potrebe, da shranimo zgladeno sliko v polni rezoluciji(razlog za spremembo velikosti).

\subsubsection{Zakaj piramida?}
\begin{itemize}
\item omogo\v{c}a u\v{c}inkovito implementacijo,
\item omogo\v{c}a ve\v{c}stopenjsko detekcijo objektov,
\item omogo\v{c}a ve\v{c}stopenjsko detekcijo robov, ...
\end{itemize}

\newpage

\section{Odvodi slik in detekcija robov}

\end{document}



















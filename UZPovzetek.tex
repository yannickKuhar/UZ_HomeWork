\documentclass[11pt]{article}

\usepackage[pdftex]{graphicx}
\usepackage{amsmath}

\title{UZ Povzetek}
\author{Yannick Kuhar}
\date{\today}

\begin{document}

\clearpage
\maketitle
\thispagestyle{empty}

\newpage

\section{Image formation}

\subsection{Film image}

\v{C}e postavimo objekt pred film ne dobimo dobre slike. Dodamo bariero, ki blokira ve\v{c}ino \v{z}arkov zmanj\v{s}a zameglitev. Odprtina, ki prepu\v{s}\v{c}a \v{z}arke se imenuje \textbf{aperture}(\textit{ang.}). \\
\\
U\v{c}inki velikosti \textbf{apertureja}:
\begin{itemize}
\item \textbf{prevelik}, prepusti preve\v{c} svetlobe kar povzro\v{c}i zameglitev slike
\item \textbf{premajhen}. prepusti premalo svetlobe kar povzro\v{c}i zameglitev slike
\end{itemize}
\
V splo\v{s}nem sta oba ekstrema slaba. Te\v{z}vo povzroca na sploh majhno stevilo zarkov, ki zadanejo film in nastane temna slika. \\
To odpravimo z dadajo le\v{c}e med objekt in film. Le\v{c}a fokusira \v{z}arke na film. To\v{c}ke na nekaterih razdalijah ostanejo fokusirane na nekatih pa zamegljene. \\
Pri uporabi tanke le\v{c}e, to\v{c}ke na razlicnih globinah fokusirajo na razlicnih globinah slikovne raznine(\textit{ang.} \textbf{image plane}). \\
\\
\textbf{Depth of field} zardalija med \textbf{slikovnim ravninam} med katerimi je u\v{c}inek zameglitve dovolj majhen. Majhni \textbf{aperture} pove\v{c}a \textbf{depth of field}. \\
\\
\textbf{Field of view}(\textbf{FOV}) ($2 \times \phi$) je kotna mera prostora, ki ga zazna kamera. \\
\\
$\phi = tan^{-1}(\frac{d}{2f})$\\
\\
\textbf{Field of view}:
\begin{itemize}
\item majhen \textbf{focal length}($f$) prinese \v{s}irokokotno sliko
\item visoko \textbf{focal length}($f$) prinese teleskopsko sliko
\end{itemize}
\
\textbf{Aberacije}:
\begin{itemize}
\item \textbf{Chromatic aberration}, razli\v{c}ne valovne dolzine se lomijo pod razli\v{c}nimi koti in fokusirajo pod razli\v{c}nimi razdalijami
\item \textbf{Spherical aberration}, sferi\v{c}ne le\v{c}e ne fokusirajo svetlobo perfektno, \v{z}arki blizje robu se fokusirajo bli\v{z}je kot \v{z}arki pri centru
\end{itemize}

\newpage

\textbf{Vignetting} se zgodi, ko imamo ve\v{c} kot eno le\v{c}o v kameri. Fokusirani \v{z}arki iz prve(proti centru) le\v{c}e se lomijo preko druge. Ker so \v{z}arki iz prve usmerjeni proti centru se namo prseslikajo preko druge(\v{z}arki pod kotom se filtriranjo). To povzro\v{c}i temne kote.\\
\\
Pogosta nepravilnost je tudi t.i. \textbf{Radial distortion}(\textit{ang.}), ki se zgodi zaradi neprefektnosti le\v{c}.

\subsection{Digital image}
Namesto filma uporabimo matriko(tabelo) senzorjev. Diskretiziramo sliko v piksle in kvantiziramo svetlobo v intenzitetne nivoje.

\subsection{CCD vs CMOS}
\begin{itemize}
\item v obeh fotoni povzro\v{c}ijo naboj na senzorjih
\item \textbf{CCD} prebere naboj po principu \textbf{FIFO} in digitalizira
\item \textbf{CMOS} digitalizira vsak piksel posebaj
\item \textbf{CCD} slike so bolj\v{s}e kvalitete
\item \textbf{CMOS} je cenej\v{s}i za izdelavo
\end{itemize}
\
Zaradi osvetljenosti, ki je dolo\v{c}ena s strani zelene barve, uporabimo dvakrat toliko zelenih senzorjev kot modrih in rde\v{c}ih. \v{C}love\v{s}ki vid je bolj ob\v{c}utljiv na spremembe intenzitete kot na barvne spremembe.

\newpage

\section{Image processing 1}
Binarne slike imajo samo dve mo\v{z}nosti intenzitete:
\begin{itemize}
\item \textbf{ospredje}(1)
\item \textbf{ozadje}(0)
\end{itemize}

\section{Image tresholding}
Spremenimo sliko v binarno masko. Spoznali smo tri na\v{c}ine:
\begin{itemize}
\item \textbf{single threshold} $F_T[i, j] = $
$
\begin{cases}
	1, & \text{if } F[i, j] \geq T \\
	0, & \text{sicer}
\end{cases}
$
\item \textbf{two thresholds} $F_T[i, j] = $
$
\begin{cases}
	1, & \text{if } T_1 \leq F[i, j] \leq T_2 \\
	0, & \text{sicer}
\end{cases}
$
\item \textbf{apply a classifier} $F_T[i, j] = $
$
\begin{cases}
	1, & \text{if } F[i, j] \in Z \\
	0, & \text{sicer}
\end{cases}
$
\end{itemize}
\
Binarne slike lahko opi\v{s}emo z \textbf{bimodalnim histogramom}(ima le dva stolpca in predstavlja porazdelitev pikslov).

\subsection{Globalna binarizacija}
Najdi tak\v{s}en $T$, ki minimizira varianco med razredi, katere $T$ lo\v{c}uje. \\
\\
$\sigma^2_{within}(T) = n_1(T)\sigma^2_1 + n_2(T)\sigma^2_2$ \\
\\
$n_1(T) = |\{I_{(x, y)} < T\}|$, $n_2(T) = |\{I_{(x, y)} \geq T\}|$

Oziroma druga\v{c}e maksimiziramo vmesen razred. \\
\\
$\sigma^2_{between}(T) = \sigma^2 - \sigma^2_{within}(T) = n_1(T)n_2(T)[\mu_1(T) - \mu_2(T)]^2$

\newpage

\subsection{Otsujev algoritem}

\end{document}


















